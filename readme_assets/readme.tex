\documentclass [a4paper,12pt]{article}

\usepackage [utf8] {inputenc}
\usepackage {amsfonts, amsthm}
\usepackage {amsmath, mathrsfs}

\setlength\parindent{0pt}

\begin {document}

\section{Forward evaluation}

Let $\mathcal{D} = \{(x, y)\}$ be a dataset of images $x$ and labels $y$. 
\\\\
Let $N(x; W, b)$ be the neural network function, with weights $W$ and biases $b$, taking inputs from a square $n \times n$ black-and-white image $x$ with pixel values in $[0,1]$ and outputting $c$ categories, that is

$$
N: [0,1]^{n \times n} \longrightarrow [0,1]^c
$$

As always this function is a composition of $d$ layers $N(x) = (N_1 \circ \cdots \circ N_d)(x)$. For each layer let
\begin{align*}
&x^k = N_k(x^{k-1}) = \sigma(W^k x^{k-1} + b^k) \qquad \forall \; k = 1, \dots, d \\
&x_0 = x \in [0,1]^{n^2} \quad \text{(the flattened image)}
\end{align*}
Here $W^k$ is a matrix of dimension $n_k \times n_{k-1}$, and $b$ a vector in $\mathbb{R}^{n_k}$. As usual the dimensions of the affine transformations $\{n_k\}_{k=0}^d$ are to choose as part of the network design process.
Therefore we will refer to $W$ and $b$ as the lists of the matrices $\{W^k\}$ and vectors $\{b^k\}$ respectively.
Furthermore we assume the activation function $\sigma$ to be the same for all layers.
\\\\
In the code, the evaluation of the neural network function $N$ is performed in the function \verb|guess|, while \verb|read| preserves and returns the vectors
$$
v^k = W^k x^{k-1} + b^k
$$
resulting from the affine transformation alone (without applying the activation $\sigma$). This extra step allows backpropagation.


\section{Backpropagation}

Again, let $\mathcal{D} = \{(x, y)\}$ be a dataset of images $x$ and corresponding labels $y \in \{0,1\}^c$ encoded as one-hot vectors. Given any loss function
$$
L: [0,1]^c \times \{0,1\}^c \longrightarrow \mathbb{R}
$$
we use $L$ to quantify the neural network's performance by evaluating $L(N(x),y)$. The derivative of this last expression with respect to each parameter $W^k, b^k$ is then used to adjust the parameters themselves through gradient descent.
\\\\
By the standard chain-rule for derivatives we get 
\begin{align*}
\partial_{W^d} L(N(x), y) &= \partial_N L(N(x),y) \; \partial_{W^d} x^d \\
\partial_{b^d} L(N(x), y) &= \partial_N L(N(x),y) \; \partial_{b^d} x^d
\end{align*}
for the last ($d$-th) level of the network. More generally, for every $0 \leq k \leq d-1$
\begin{equation} \label{network-derivative}
\begin{aligned}
\partial_{W^{k}} L(N(x), y) &= \partial_N L(N(x),y) \left[ \; \prod_{h=0}^{d-k-1} \partial_{x^{d-h-1}} x^{d-h} \right] \partial_{W^{k}} x^{k}
\\
\partial_{b^{k}} L(N(x), y) &= \partial_N L(N(x),y) \left[ \; \prod_{h=0}^{d-k-1} \partial_{x^{d-h-1}} x^{d-h} \right] \partial_{b^{k}} x^{k}
\end{aligned}
\end{equation}
Of course
$$
\partial_{x^{k-1}} x^k = \textit{diag}( \sigma ' (W^k x^{k-1} + b^k) ) \; W^k
$$
where $\textit{diag}(v)$ is the diagonal matrix having the vector $v$ as diagonal.
Instead $\partial_{W^{k}} x^k$ are tensors of rank 3 which we prefer to write component-wise:
$$
\partial_{W^{k}_{ij}} x^k_t = \sigma ' (v^k_t) x^{k-1}_j \delta_{it}
$$
while
$$
\partial_{b^{k}_{i}} x^k_t = \sigma ' (v^k_t) x^{k-1}_j \delta_{it}
$$
The vectors $v^k = W^k x^{k-1} + b^k$ were returned from the \verb|read| function.

The function \verb|gradErr| computes $\partial_{W^{k}} L(N(x),y)$ and $\partial_{b^{k}} L(N(x),y)$ going backwards from $k = d$ to $k = 1$ and stores them in two separate lists, reverses them, and returns them.
\end{document}